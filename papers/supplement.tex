
% ============================================================================
\documentclass{article}

% ============================================================================
\usepackage{amsmath}
\usepackage{amssymb}
\usepackage{amsfonts}
\usepackage[
 top=1in,
 bottom=1.5in,
 inner=1.5in,
 outer=1.5in]{geometry}
\usepackage[T1]{fontenc}
\usepackage{parskip}
\usepackage{color}
\usepackage{soul}

% ============================================================================
\newcommand{\E}[2]{\mathbb{E}_{#2}\left[#1\right]}
\newcommand{\m}[1]{\boldsymbol{#1}}

\DeclareMathOperator{\trace}{tr}

% ============================================================================
\begin{document}

\author{Bradley Worley}
\title{Supplementary information for: \\
\emph{\Large Variationally reweighted least squares for sparse recovery}}

\maketitle

% ============================================================================
\section{IRLS}
\label{s:irls}
The EC-IRLS and UC-IRLS algorithms follow from minimizing the function,
\begin{equation}
\psi(\m{x}, \m{\sigma}) =
 \frac{1}{2} \sum_{i=1}^n \left(
  \frac{|x_i|^2}{\sigma_i^2} + \frac{\sigma_i^2}{\xi^2}
 \right)
\label{eq:psi}
\end{equation}
with and without data agreement constraints on $\m{x}$, respectively.
Iterations consist of alternating $\m{x}$ and $\m{\sigma}$ updates.
The iterates are initialized to $\m{x}^{(0)} = \m{0}$ and
$\m{\sigma}^{(0)} = \m{1}$. To ease notation, define
the (diagonal) IRLS weight matrix $\m{W}$ such that,
\begin{equation}
W_{ij} = \begin{cases}
 1 / \sigma_i^2 &\text{if } i = j \\
 0 &\text{if } i \ne j
\end{cases}
\label{eq:wmatrix}
\end{equation}

% ----------------------------------------------------------------------------
\subsection{Equality-constrained IRLS}
\subsubsection{Updates to $\m{x}$}
The EC-IRLS $\m{x}$-update is given by the sub-problem,
\begin{equation}
\m{x}^{(t+1)} =
\left\{
\begin{aligned}
 \underset{\m{x}}{\arg\min} &\;
 \psi\big( \m{x}, \m{\sigma}^{(t)} \big)
\\
 \text{such that} &\; \m{y} = \m{A} \m{x}
\end{aligned}
\right.
\label{prob:ec_irls_x}
\end{equation}

Introduce the Lagrangian for \eqref{prob:ec_irls_x},
\begin{equation}
\begin{aligned}
\mathcal{L}(\m{x}, \m{\lambda}) &=
 \psi\big( \m{x}, \m{\sigma}^{(t)} \big) +
 \m{\lambda}^* ( \m{y} - \m{A} \m{x} )
\\ &=
 \frac{1}{2} \m{x}^* \m{W}^{(t)} \m{x} +
 \m{\lambda}^* ( \m{y} - \m{A} \m{x} )
\end{aligned}
\label{eq:lagrangian_irls_x}
\end{equation}
where terms that do not involve $\m{x}$ have been dropped, because
they contribute a constant offset to the Lagrangian. Setting the
$\m{x}$-gradient of $\mathcal{L}(\m{x}, \m{\lambda})$ equal to
zero yields,
\begin{equation}
\begin{aligned}
\nabla_{\m{x}} \mathcal{L}(\m{x}, \m{\lambda}) &=
 \m{W}^{(t)} \m{x} - \m{A}^* \m{\lambda} = \m{0}
\\ \implies\quad
\m{x} &= \big(\m{W}^{(t)}\big)^{-1} \m{A}^* \m{\lambda}
\end{aligned}
\label{eq:ec_irls_xl}
\end{equation}

Substituting this back into the Lagrangian yields the dual function,
\begin{equation}
\begin{aligned}
g(\m{\lambda}) &\triangleq
 \underset{\m{x}}{\inf} \;\mathcal{L}(\m{x}, \m{\lambda})
\\ &=
 \frac{1}{2} \m{\lambda}^* \m{A}
 \big(\m{W}^{(t)}\big)^{-1} \m{W}^{(t)}
 \big(\m{W}^{(t)}\big)^{-1} \m{A}^* \m{\lambda}
 +\m{\lambda}^* \m{y}
 -\m{\lambda}^* \m{A} \big(\m{W}^{(t)}\big)^{-1} \m{A}^* \m{\lambda}
\\ &=
 -\frac{1}{2} \m{\lambda}^* \m{A}
  \big(\m{W}^{(t)}\big)^{-1}
  \m{A}^* \m{\lambda}
 +\m{\lambda}^* \m{y}
\end{aligned}
\label{eq:ec_irls_dual}
\end{equation}

Maximizing this dual function by setting its $\m{\lambda}$-gradient
to zero yields,
\begin{equation}
\begin{aligned}
\nabla_{\m{\lambda}} g(\m{\lambda}) &=
 -\m{A} \big(\m{W}^{(t)}\big)^{-1} \m{A}^* \m{\lambda}
 +\m{y} = \m{0}
\\ \implies \quad
\m{\lambda} &=
 \left( \m{A} \big(\m{W}^{(t)}\big)^{-1} \m{A}^* \right)^{-1} \m{y}
\end{aligned}
\label{eq:ec_irls_lambda}
\end{equation}

Thus, the solution to sub-problem \eqref{prob:ec_irls_x} is,
\begin{equation}
\m{x}^{(t+1)} =
 \big(\m{W}^{(t)}\big)^{-1} \m{A}^*
 \left( \m{A} \big(\m{W}^{(t)}\big)^{-1} \m{A}^* \right)^{-1} \m{y}
\label{eq:ec_irls_x}
\end{equation}

% ----------------------------------------------------------------------------
\subsubsection{Updates to $\m{\sigma}$}
The EC-IRLS $\m{\sigma}$-update is given by the sub-problem,
\begin{equation}
\m{\sigma}^{(t+1)} =
 \underset{\m{\sigma}}{\arg\min} \;
 \psi\big( \m{x}^{(t+1)}, \m{\sigma} \big)
\label{prob:ec_irls_sigma}
\end{equation}

Because $\psi(\m{x}, \m{\sigma})$ is separable with respect to each
$\sigma_i$, this involves the solution of $n$ independent sub-problems:
\begin{equation}
\sigma_i^{(t+1)} =
 \underset{\sigma_i}{\arg\min} \left(
  \frac{\big| x_i^{(t+1)} \big|^2}{2 \sigma_i^2} +
  \frac{\sigma_i^2}{2 \xi^2}
 \right)
\label{prob:ec_irls_sigma_i}
\end{equation}

Setting the gradient of this function equal to zero yields the
solution,
\begin{equation}
\begin{aligned}
\frac{\partial}{\partial \sigma_i} \left(
  \frac{\big| x_i^{(t+1)} \big|^2}{2 \sigma_i^2} +
  \frac{\sigma_i^2}{2 \xi^2}
 \right) &=
-\frac{\big| x_i^{(t+1)} \big|^2}{\sigma_i^3} +
 \frac{\sigma_i}{\xi^2} = 0
\\ \implies \quad
\sigma_i^{(t+1)} &= \sqrt{\xi \big| x_i^{(t+1)} \big|}
\end{aligned}
\label{eq:ec_irls_sigma_i}
\end{equation}
which is equivalent (by the definition of $\m{W}$) to the following
weight update:
\begin{equation}
W_{ii}^{(t+1)} =
 \frac{1}{\big( \sigma_i^{(t+1)} \big)^2} =
 \frac{1}{\xi \big| x_i^{(t+1)} \big|}
\label{eq:ec_irls_w}
\end{equation}

% ----------------------------------------------------------------------------
\subsection{Unconstrained IRLS}
In UC-IRLS, the exact constraints on $\m{x}$ are replaced with an
$\ell_2$-norm term, resulting in a new objective function,
\begin{equation}
\psi_\nu(\m{x}, \m{\sigma}) =
 \psi(\m{x}, \m{\sigma}) + \frac{1}{2 \nu} \| \m{y} - \m{A} \m{x} \|_2^2
\label{eq:uc_irls_obj}
\end{equation}

% ----------------------------------------------------------------------------
\subsubsection{Updates to $\m{x}$}
The UC-IRLS $\m{x}$-update is given by the sub-problem,
\begin{equation}
\m{x}^{(t+1)} =
 \underset{\m{x}}{\arg\min} \;
 \psi_\nu\big( \m{x}, \m{\sigma}^{(t)} \big)
\label{prob:uc_irls_x}
\end{equation}

Setting the $\m{x}$-gradient of
$\psi_\nu\big( \m{x}, \m{\sigma}^{(t)} \big)$
equal to zero yields the solution to sub-problem \eqref{prob:uc_irls_x},
\begin{equation}
\begin{aligned}
\nabla_{\m{x}} \psi_\nu\big( \m{x}, \m{\sigma}^{(t)} \big) &=
 \m{W}^{(t)} \m{x} + \nu^{-1} \m{A}^* \m{A} \m{x} -
 \nu^{-1} \m{A}^* \m{y} = \m{0}
\\ \implies \quad
\m{x}^{(t+1)} &=
 \nu^{-1} \left(
  \m{W}^{(t)} + \nu^{-1} \m{A}^* \m{A}
 \right)^{-1} \m{A}^* \m{y}
\end{aligned}
\label{eq:uc_irls_x}
\end{equation}

% ----------------------------------------------------------------------------
\subsubsection{Updates to $\m{\sigma}$}
Because $\psi_\nu(\m{x}, \m{\sigma})$ only depends on $\m{\sigma}$ through
$\psi(\m{x}, \m{\sigma})$, the UC-IRLS $\m{\sigma}$-update is identical
to the EC-IRLS update given by equation \eqref{eq:ec_irls_sigma_i}.

% ============================================================================
\section{Direct VRLS}
\label{s:vrls}
Direct VRLS algorithms involve minimizing the function,
\begin{equation}
\bar\psi(\m{\mu}, \m{\gamma}, \m{\alpha}, \m{\beta}) \triangleq
\E{\psi(\m{x}, \m{\sigma})}{q} =
 \frac{1}{2} \sum_{i=1}^n \left(
  \frac{|\mu_i|^2 + \gamma_i}{\beta_i \sqrt{\alpha_i}} +
  \frac{\beta_i (\sqrt{\alpha_i} + \beta_i)}{\xi^2}
 \right)
\label{eq:psibar}
\end{equation}
with and without data agreement constraints on $\{\m{\mu}, \m{\gamma}\}$,
respectively, where $q$ denotes the variational approximation of the
posterior $p(\m{x}\mid\m{y})$.

\hl{FIXME: iterate initialization}

To ease notation once again, define
the (diagonal) VRLS weight matrix $\m{V}$ such that,
\begin{equation}
V_{ij} = \begin{cases}
 \E{1 / \sigma_i^2}{q} &\text{if } i = j \\
 0 &\text{if } i \ne j
\end{cases}
\label{eq:vmatrix}
\end{equation}

\hl{FIXME}

% ============================================================================
\section{The $h\ell_1$ density}
\label{s:hprior}
The ``hierarchical-form Laplace'' probability density function,
or $h\ell_1$ density for short, has the following form:
\begin{equation}
h(\sigma; \alpha, \beta) =
 \frac{1}{Z(\alpha, \beta)}
 \exp\left\{
  -\frac{\alpha}{2 \sigma^2}
  -\frac{\sigma^2}{2 \beta^2}
 \right\}
\label{eq:hprior}
\end{equation}
where $\alpha, \beta, \sigma > 0$ and $Z(\alpha, \beta)$ is the
normalization function, which is defined to be,
\begin{equation}
\begin{aligned}
Z(\alpha, \beta) &\triangleq
 \int_0^\infty \exp\left\{
  -\frac{\alpha}{2 \sigma^2}
  -\frac{\sigma^2}{2 \beta^2}
 \right\} d\sigma
\\ &=
 \beta \sqrt{\frac{\pi}{2}}
 \exp\left\{ -\frac{\sqrt{\alpha}}{\beta} \right\}
\end{aligned}
\label{eq:hprior_z}
\end{equation}

Thus, the fully normalized $h\ell_1$ density is equal to,
\begin{equation}
h(\sigma; \alpha, \beta) =
 \frac{1}{\beta} \sqrt{\frac{2}{\pi}}
 \exp\left\{
  -\frac{\alpha}{2 \sigma^2}
  -\frac{\sigma^2}{2 \beta^2}
  +\frac{\sqrt{\alpha}}{\beta}
 \right\}
\label{eq:hprior_norm}
\end{equation}

% ----------------------------------------------------------------------------
\subsection{Relevant expectations}
\subsubsection{Expected variance}
The expected value of $\sigma^2$ is given by,
\begin{equation}
\E{\sigma^2}{h} =
 \int_0^\infty \sigma^2 h(\sigma; \alpha, \beta) \, d\sigma =
 \beta (\sqrt{\alpha} + \beta)
\label{eq:expect_s2}
\end{equation}

\subsubsection{Expected inverse variance}
The expected value of $1 / \sigma^2$ is given by,
\begin{equation}
\E{\frac{1}{\sigma^2}}{h} =
 \int_0^\infty \frac{1}{\sigma^2} h(\sigma; \alpha, \beta) \, d\sigma =
 \frac{1}{\beta \sqrt{\alpha}}
\label{eq:expect_s2_inv}
\end{equation}

\subsubsection{Entropy}
Using \eqref{eq:expect_s2} and \eqref{eq:expect_s2_inv}, the entropy
of the $h\ell_1$ density is found to equal,
\begin{equation}
\mathbb{H}[h(\sigma; \alpha, \beta)] =
 -\E{\ln h(\sigma; \alpha, \beta)}{h} =
 \frac{1}{2} + \frac{1}{2} \ln\left( \frac{\pi \beta^2}{2} \right)
\label{eq:hprior_entropy}
\end{equation}

% ----------------------------------------------------------------------------
\subsection{Relationship to the Laplace density}
The probability density function of a Laplace-distributed random variable
$x \sim \mathcal{L}(\xi)$ is given by,
\begin{equation}
f_{\mathcal{L}}(x; \xi) =
 \frac{1}{2 \xi} \exp\left\{
  -\frac{|x|}{\xi}
 \right\}
\label{eq:laplace_pdf}
\end{equation}
where $\xi > 0$ is called the scale parameter. This density may be
obtained by marginalizing a suitably constructed joint density
involving $x$. More concretely, consider the joint probability
density of the random variables $x$ and $\sigma$, where:
\begin{itemize}
 \item $x \mid \sigma \sim \mathcal{N}(0, \sigma^2)$, i.e.\
  conditioned on $\sigma$, $x$ is normally distributed with
  zero mean and variance $\sigma^2$, and
 \item $\sigma \sim \mathcal{R}(\xi)$, i.e.\
  $\sigma$ is Rayleigh-distributed with scale parameter $\xi$.
\end{itemize}

The joint density over $(x,\sigma)$ is equal to,
\begin{equation}
\begin{aligned}
p(x, \sigma) &=
 p(x \mid \sigma) \, p(\sigma)
\\ &=
 f_{\mathcal{N}}(x; 0, \sigma^2) \,
 f_{\mathcal{R}}(\sigma; \xi)
\\ &=
 \frac{1}{\sqrt{2 \pi \sigma^2}} \exp\left\{
  -\frac{|x|^2}{2 \sigma^2}
 \right\} \cdot
 \frac{\sigma}{\xi^2} \exp\left\{
  -\frac{\sigma^2}{2 \xi^2}
 \right\}
\\ &=
 \frac{1}{\xi^2 \sqrt{2 \pi}} \exp\left\{
  -\frac{|x|^2}{2 \sigma^2}
  -\frac{\sigma^2}{\xi^2}
 \right\}
\end{aligned}
\label{eq:joint_pdf}
\end{equation}
which is proportional to an $h\ell_1$ density having parameters
$\alpha = |x|^2$ and $\beta = \xi$. Marginalizing this joint density
with respect to $\sigma$ using \eqref{eq:hprior_z} results in,
\begin{equation}
\begin{aligned}
p(x) =
\int_0^\infty p(x, \sigma) \, d\sigma &=
 \frac{1}{\xi^2 \sqrt{2 \pi}} \int_0^\infty \exp\left\{
  -\frac{|x|^2}{2 \sigma^2}
  -\frac{\sigma^2}{\xi^2}
 \right\} d\sigma
\\ &=
 \frac{1}{\xi^2 \sqrt{2 \pi}} Z(|x|^2, \xi)
\\ &=
 \frac{1}{\xi^2 \sqrt{2 \pi}}
 \xi \sqrt{\frac{\pi}{2}}
 \exp\left\{ -\frac{\sqrt{|x|^2}}{\xi} \right\}
\\ &=
 \frac{1}{2 \xi} \exp\left\{ -\frac{|x|}{\xi} \right\}
\end{aligned}
\label{eq:marginal_pdf}
\end{equation}
which is a Laplace distribution on $x$ with shape parameter $\xi$.

% ============================================================================
\end{document}
